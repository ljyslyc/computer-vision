\documentclass[withoutpreface,bwprint]{cumcmthesis} %去掉封面与编号页
\usepackage[final]{pdfpages}
\title{《软件需求工程 B》 大作业课题}
\tihao{A}

\baominghao{4321}
\schoolname{武汉理工大学}
\membera{小米}
\memberb{向左}
\memberc{哈哈}
\supervisor{老师}
\yearinput{2017}
\monthinput{08}
\dayinput{22}

% 添加pdf封面
% \usepackage[final]{pdfpages}
% \includepdf[pages={1,2}]{head.pdf}
% \newpage


% \begin{enumerate}[itemindent=1em]  %在这里设置缩进的距离
%     \item 将信号量S的值减1,即进行S = S-1;   
% \end{enumerate}

% \newpage
% % 代码附录
% \begin{appendices}
% \section{爬取店铺POI}
% \begin{lstlisting}[language=python]
%  \end{lstlisting}
% \section{评论数据爬取}
% \begin{lstlisting}[language=python]
% \end{lstlisting}
% \end{appendices}

\begin{document}
\includepdf{exp_head.pdf}
\newpage
% \maketitle
%  \begin{abstract}
 
% \keywords{\TeX{}\quad  图片\quad   表格\quad  公式}
% \end{abstract}
% \includepdf[pages={1,2}]{head.pdf} 
% \newpage
% 目录
% \tableofcontents

\section*{\LARGE 实验一:混合图像}

\section{实验背景}

本次实验的目标是使用Oliva、Torralba和Schyns在SIGGRAPH 2006论文中描述的方法创建混合图像。混合图像是一种合成的静态图像,观众对其内容的理解随视距的变化而变化。混合图像的基本想法是,当距离图像较近时,图像中的高频信号更容易被观众所感知,但距离图像较远时,观众更容易感知到图像中的低频(平滑)部分。通过将一幅图像的高频部分与另一幅图像的低频部分混合,你就可以得到一幅混合图像,在不同的距离上会得到不同的内容解释。
\section{实验目标}
本次实验的目的是利用图像处理技术,分别利用低频与高频滤波器对图像进行处理,从而掌握卷积操作、高斯滤波器、图像合成等不同的基础处理方法。

\section{问题分析及设计思路}


\section{解决方案及结果分析}

\section{优化方案及对比}


\section{实验改进与展望}


\newpage

\section*{\LARGE 实验二:图像融合}

\section{实验背景}
\section{实验目标}
使用多分辨率融合技术无缝地融合两幅图像,图像通过轻微的变形和平滑的接缝将两个图像连接在一起。本次实验帮助学生掌握高斯金字塔、拉普拉斯金字塔以及多分辨率图像还原等技术处理过程。
\section{设计思想及原理}

\section{问题分析及设计思路}


\section{解决方案及结果分析}

\section{优化方案及对比}


\section{实验改进与展望}



\section{课程总结}
本次实验学习了设计到在linux系统上多个c++文件的编译,因此学习了如何

% \newpage
% % 代码附录
\begin{appendices}
\section{实验二:彩色图像修复}
\begin{lstlisting}[language=python]
   
\end{lstlisting}
\end{appendices}
%参考文献
\begin{thebibliography}{9}%宽度9
    \bibitem[1]{1}
    李洁, 张瑜慧. 信号量在生产者-消费者及其变形问题中的应用[J]. 福建电脑, 2012(02):175-177.
    \bibitem[2]{2}
    李志民, 赵一丁, 底恒. 操作系统进程同步的教学实践[C]// 计算机研究新进展(2010)——河南省计算机学会2010年学术年会论文集. 2010.步的教学实践[C]计算机研究新进展(2010)——河南省计算机学会2010年学术年会论文集. 2010.
    \bibitem[3]{3}
    陈涛,任海兰. 基于Linux的多线程池并发Web服务器设计[J]. 电子设计工程, 2015(11):175-177.
    \bibitem[4]{4}
    李盼盼, 赵浩. 基于信号量机制的生产者消费者问题的分析[J]. 无线互联科技, 2013(11):101-102.
\end{thebibliography}

\end{document}
